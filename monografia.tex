%%%%%%%%%%%%%%%%%%%%%%%%%%%%%%%%%%%%%%%%
% Classe do documento
%%%%%%%%%%%%%%%%%%%%%%%%%%%%%%%%%%%%%%%%

% Opções:
%  - Graduação: bacharelado|engenharia|licenciatura
%  - Pós-graduação: [qualificacao], mestrado|doutorado, ppca|ppginf

% \documentclass[engenharia]{UnB-CIC}%
\documentclass[qualificacao,mestrado,ppginf]{UnB-CIC}%

\usepackage{pdfpages}% incluir PDFs, usado no apêndice
%\usepackage{algorithm}
%\usepackage[noend]{algpseudocode}
\usepackage[linesnumbered,ruled]{algorithm2e}
\usepackage{amsmath, amssymb}
\usepackage{amsthm}
\usepackage{subcaption}

%tikz packages
\usepackage{tikz}
\usetikzlibrary{trees,er,matrix,mindmap,backgrounds}
\usetikzlibrary{shadings,intersections,calc}
\usetikzlibrary{shapes,arrows,patterns,snakes}
\usetikzlibrary{automata,shadows,fit,fadings,matrix,positioning}
\usepackage{pgfplots}
\usepackage{pgfkeys}
\usepackage{mathrsfs}
\usepackage{multirow}
\usepgfplotslibrary{units}
\pgfdeclarelayer{nodelayer}
\pgfdeclarelayer{edgelayer}
\pgfsetlayers{edgelayer,nodelayer,main}


%%%%%%%%%%%%%%%%%%%%%%%%%%%%%%%%%%%%%%%%
% Informações do Trabalho
%%%%%%%%%%%%%%%%%%%%%%%%%%%%%%%%%%%%%%%%
\orientador[a]{\prof[a] \dr[a] Cl\'audia Nalon}{CIC/UnB}%
%\coorientador{\prof \dr José Ralha}{CIC/UnB}
\coordenador{\prof \dr Bruno Luiggi Macchiavello Espinoza}{CIC/UnB}%
\diamesano{}{}{2017}%

\membrobanca{\prof \dr }{}%
\membrobanca{\dr }{}%

\autor{Daniella}{Angelos}%

\titulo{Combined Proof Methods for Multimodal Logic}%

\palavraschave{l\'ogicas modais, resolução, sat-solvers}%
\keywords{modal logics, resolution, sat-solvers}%

\newcommand{\unbcic}{\texttt{UnB-CIC}}%

%%%%%%%%%%%%%%%%%%%%%%%%%%%%%%%%%%%%%%%%
% Texto
%%%%%%%%%%%%%%%%%%%%%%%%%%%%%%%%%%%%%%%%
\begin{document}%
\newcommand{\then}{\Rightarrow}
\renewcommand{\iff}{\Leftrightarrow}


\newcommand{\agent}{\ensuremath{a}}
\newcommand{\Agents}{\ensuremath{{\mathcal{A}}}}
\newcommand{\Prop}{\ensuremath{{\mathcal{P}}}}
\newcommand{\wff}{\WFF{\system{K}{n}{}}{}}
\newcommand{\wffml}{\ifmmode\text{WFF}^{\scriptsize ml}_{\scriptsize \system{K}{n}{}}}
\newcommand{\sat}[3]{\ensuremath{\langle #1, #2 \rangle \models #3}}
\newcommand{\Model}{\ensuremath{\mathcal{M}}}
\newcommand{\formula}{\ensuremath{\varphi}}
\newcommand{\trule}{\ensuremath{\sigma}}
\newcommand{\calculus}[1]{\ensuremath{\mathcal{C}_{#1}}}
\newcommand{\Literals}{\ensuremath{\mathcal{L}}}
\newcommand{\ml}{\ensuremath{i}}
\newcommand{\cprop}[2]{\ensuremath{\ml: \bigvee^{#1}_{#2 = 1}l_{#2}}}
\newcommand{\cneg}{\ensuremath{\ml: l \then \pos{a}m}}
\newcommand{\cpos}{\ensuremath{\ml: l \then \nec{a}m}}
\renewcommand{\stackrel}[1]{\ensuremath{\overset{\text{#1}}{=}}}
\newcommand{\ckn}{\ensuremath{\cal{C}_{\system{K}{n}{}}}}
\newcommand{\ex}{\ensuremath{\nec{}(p \then \pos{})}}

\newtheorem{theorem}{Theorem}[section]
\newtheorem{lemma}[theorem]{Lemma}
%\newtheorem{proposition}[theorem]{Proposition}
%\newtheorem{corollary}[theorem]{Corollary}
%\newtheorem{definition}[theorem]{Definition}

%\newenvironment{proof}[1][Proof]{\begin{trivlist}
%\item[\hskip \labelsep {\bfseries #1}]}{\end{trivlist}}
%\newenvironment{example}[1][Example]{\begin{trivlist}
%\item[\hskip \labelsep {\bfseries #1}]}{\end{trivlist}}
%\newenvironment{remark}[1][Remark]{\begin{trivlist}
%\item[\hskip \labelsep {\bfseries #1}]}{\end{trivlist}}

\newcounter{example}[section]
\newenvironment{example}[1][]{\refstepcounter{example}\par\medskip
   \noindent \textbf{Example~\theexample #1} \rmfamily}{\medskip}


\ifx\fmtname\@psfmtname \else \def\cmsy@{2}\fi % make sure we get cmsy
\def\sometime{\mathord{\hbox{\normalsize$\mathchar"0\cmsy@7D$}}}


\newcommand{\always}{\raisebox{-.2ex}{
           \mbox{\unitlength=0.9ex
           \begin{picture}(2.3,2.3)
           \linethickness{0.06ex}
           \put(0,0){\line(1,0){2.3}}
           \put(0,2.3){\line(1,0){2.3}}
           \put(0,0){\line(0,1){2.3}}
           \put(2.3,0){\line(0,1){2.3}}
           \end{picture}}}
          \,}

\newcommand{\alwaysi}[1]{\raisebox{-.2ex}{
           \mbox{\unitlength=0.9ex
           \begin{picture}(2,2)
			   \linethickness{0.06ex}
			   \put(0,0){\line(1,0){2}}
			   \put(0,2){\line(1,0){2}}
			   \put(0,0){\line(0,1){2}}
			   \put(2,0){\line(0,1){2}}
                           \put(0.5,0.5){\scriptsize$#1$}
			   \end{picture}}}}

\newcommand{\nec}[1]{\!\alwaysi{#1}\,}
\newcommand{\pos}[1]{{	   \mbox{\unitlength=0.8ex
			   \begin{picture}(2,2)
			   \linethickness{0.06ex}
			   \put(0,0){$\sometime$}
                           \put(0.6,0.4){\tiny$#1$}
			   \end{picture}}}\,}

%\newcommand{\nec}[1]{{	   \mbox{\unitlength=1.5ex
			   %\begin{picture}(2,2)
			   %\linethickness{0.06ex}
			   %\put(0,0){$\always$}
                           %\put(0.8,0.4){\footnotesize$ #1$}
			   %\end{picture}}}\,}

%\newcommand{\pos}[1]{{	   \mbox{\unitlength=1.5ex
			   %\begin{picture}(1.5,1.5)
			   %\linethickness{0.06ex}
			   %\put(0,0){$\sometime$}
                           %\put(0.6,0.4){\footnotesize$#1$}
			   %\end{picture}}}\,}

%\newcommand{\pos}[1]{\sometime _{#1}}
\newcommand{\onlyif}{\Leftarrow}
\newcommand{\ifonlyif}{\Leftrightarrow}
\newcommand{\tvalue}[1]{\mbox{\it #1\/}}
\newcommand{\constant}[1]{\mbox{\rm\bf #1}}
\newcommand{\system}[3]{\raisebox{.2ex}[1.2ex]{\raisebox{-.2ex}{{\sf #1}}{$_{#2}^{#3}$}}}
%\newcommand{\system}[3]{{\sf #1}$_{#2}^{#3}$}
\newcommand{\set}[1]{\mbox{$\mathcal{#1}$}}
\newcommand{\WFF}[2]{{\sf WFF}{\mbox{$_{\mbox{\small\sf #1}^{#2}}$}}}
\newcommand{\know}[1]{\mbox{K$_{#1}\,$}}
\newcommand{\knownot}[1]{\mbox{$\nec{#1}\neg\,$}}
\newcommand{\believe}[1]{\mbox{B$_{#1}\,$}}
\newcommand{\relation}[2]{\mbox{$\mathcal #1$$_{#2}$}}
%\newcommand{\universal}{\always^{*}}
\newcommand{\universal}{\always^*}
\newcommand{\snf}[1]{{\sf SNF}\mbox{$_{\mbox{\scriptsize #1}}$}}
\newcommand{\binomial}[2]{\left(\! \begin{array}{c}
                                 #1\\
                                 #2
                                 \end{array}
                          \!\right )}
\newcommand{\ap}[1]{\alpha(#1)}

\newcommand{\ib}{\bf}

\newcommand{\NKN}{\hbox{\it NKN}}
\newcommand{\NEW}{\hbox{\it NEW}}


\newcommand{\Nat}{\mbox{$\mathbb N$}}

%\newcommand{\comment}[1]{}
     
\newcommand{\cfalse}{\constant{false}}
\newcommand{\ctrue}{\constant{true}}
\newcommand{\st}{\ensuremath{w}}
\newcommand{\St}{\ensuremath{W}}
\newcommand{\depth}{\sf profundidade}
\newcommand{\model}[1]{{\cal #1}}


    \capitulo{1_Introduction}{Introduction}%
    \capitulo{2_Language}{Modal Logics}%
    \capitulo{3_Resolution}{Modal-Layered Resolution}%
    \capitulo{4_Sat}{Satisfiability Solvers}%

    %\apendice{Apendice_Fichamento}{Fichamento de Artigo Científico}%
    %\anexo{Anexo1}{Documentação Original \unbcic\ (parcial)}%
\end{document}%
