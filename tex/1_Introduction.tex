%\section{}%
Several aspects of complex computational systems can be modelled by means of
logic languages, allowing for the characterization of notions such as
probabilities, possibilities, time, knowledge and
belief~\cite{FHMV95,Hai82,HMM83,rao:91c}. Modal logics, more specifically, have
been widely studied in Computer Science as they can naturally model, for
instance, notions of knowledge and belief in multiagent
systems~\cite{bratman1987intention,FHMV95,rao:91c}, temporal aspects in the
formal verification of problems related to concurrent and distributed
systems~\cite{Hai82,HMM83}. Modal languages are simple yet expressive languages
for reasoning over relations between different contexts or
interpretations~\cite{blackburn2002modal}.

Once a system has been specified in a logic language, it is possible to use
proof methods to verify properties of such system. In general, if $\mathcal{I}$
is the formulae set representing the implementation, and $\Formulae$ is the set
that characterizes the specification, the verification process consists in
showing that is possible to \emph{derive} the specified aspects in $\Formulae$
from $\mathcal{I}$. Each proof system defines a particular approach to deal with
formulae. Furthermore, it is possible to combine proof methods with the ultimate
goal to benefit from each method's advantages. 

In the literature, we find several successful combinations of proof systems
(e.g.~\cite{hylores, ghilardi2017interpolation,
Goetzmann+Kaminski+Smolka@ENTCS2010, gutierrez2016metric}). It is crucial to
make sure that sufficient information is shared between the combined methods.
Furthermore, combination of logic languages may increase the computational
complexity of the problem, or even lead to the undecidability of the resultant
logic satisfiability~\cite{mdml}. Therefore, it is essential to develop
reliable techniques that allow the uniform implementation of combined systems
for such combination of logic languages.

The satisfiability problem for the basic multimomodal logic is proven to be
PSPACE-complete~\cite{Spaan:coml}, for local satisfiability, and EXPTIME, for
global satisfiability~\cite{Spaan:coml}. In~\cite{nalon2015modal}, a complete
calculus based on modal levels, for both local and global reasoning for the
multimodal basic propositional modal logic, is presented. The strategies
introduced in~\cite{nalon2015modal} helps to reduce the search space for a proof
and also reduces the number of inference rules needed in the calculus. The
implementation of theses strategies showed that the partitioning in modal levels
may decrease the time spent during proofs~\cite{Nalon2016}. 

In the literature, there are several other theorem provers for modal
logics~\cite{hylores,eprover,spass,Vampire}. In this work, we focus on
\ksp~\cite{Nalon2016}, a theorem prover for the modal language
\system{K}{n}{}, which implements the clausal resolution method proposed
in~\cite{nalon2015modal}. Clauses are labelled by the modal level at which they
occur, helping to restrict unnecessary applications of the resolution inference
rules. Several refinements and simplification techniques in order to reduce the
search space for a proof are implemented. To get the best performance for a
particular formula, or class of formulae, it is important to choose the right
strategy and optimizations. 

According to~\cite{Nalon2016}, \ksp~performs well if the set of propositional
symbols are uniformly distributed over the modal levels. However, when there is
a high number of propositional symbols in just one particular level, the
performance deteriorates. One reason is that the specific normal form used
always generates satisfiable sets of propositional clauses (i.e.\ clauses
without modal operators). As resolution relies on saturation, this can be very
time consuming. We are currently investigating the use of other tools in order
to speed up the saturation process. 

Boolean Satisfiability Solvers (SAT solvers), for instance, can often solve hard
structured problems with over a million variables and several million
constraints~\cite{satchapter}. We believe that we can take advantage of the
theoretical and practical efforts that have been directed in improving the
efficiency of such solvers. 

%clause learning
One of the main reasons for the widespread use of SAT in many applications is
that solvers based on clause learning are very effective in
practice~\cite{satchapter}. The main idea behind Conflict-Driven Clause Learning
(CDCL) is to cache ``causes of conflict'' as learned clauses, and utilize this
information to prune the search in a different part of the search space
encountered later. Since their inception in the mid-90s, CDCL SAT solvers have
been applied, in many cases with remarkable success, to a number of practical
applications~\cite{cdclchapter}. 

Our implementation, which is work in progress, uses a SAT solver based on clause
learning by conflict analysis. We feed this solver with the satisfiable sets of
clauses generated by the translation procedure. Each time it identifies a
conflict in these sets due to unit propagation from some variable assignment,
one or more new clauses are learnt from the conflict analysis procedure, which
analyses the structure of unit propagation and decides which literals to include
in the new clauses~\cite{cdclchapter}. As mentioned before, as we already know
that these sets are satisfiable, we are not particularly interested in the model
generated by the SAT solver. 

We feed \ksp\ back the new clauses, hoping it can find more applications of
inference rules that deal with modal reasoning using these clauses. We believe
that by carefully choosing the set of clauses we feed the SAT solver, and
appropriately using the learnt clauses generated, we may be able to reduce the
time \ksp~spends during saturation. 

%organization of work
The rest of this work is organized as follows: Chapter~\ref{2_Language}
introduces the propositional modal language which is the main focus of this
work: \system{K}{n}{}, as well as its syntax and semantics;
Chapter~\ref{3_Resolution} and Chapter~\ref{4_Sat}, presents, respectively, a
brief overview of resolution based systems and modern SAT solvers. Finally,
in Chapter~\ref{5_Discussion} we discuss how we pretend to combine \ksp\ with a
SAT solver, and what we expect from this combination.
