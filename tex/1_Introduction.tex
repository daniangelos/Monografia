%\section{}%
Several aspects of complex computational systems can be modelled by means of
logic languages, for allowing the characterization of notions such as
probabilities, possibilities, time, knowledge and
belief~\cite{FHMV95,HMM83,Hai82,rao:91c}. Modal logics, more specifically, have
been widely studied in Computer Science as they can naturally model, for
instance, notions of knowledge and belief in multiagent
systems~\cite{bratman1987intention,FHMV95,rao:91c}, temporal aspects in the
formal verification of problems related to concurrent and distributed
systems~\cite{HMM83,Hai82}. These modalities induce a relevant gain in
expressiveness to modal languages, when compared to classical logics. Different
modalities define different modal logics.

Once a system has been specified in a logic language, it is possible to use
proof methods to verify properties of such system. In general, if $\mathcal{I}$
is the formulae set representing the implementation, and $\Formulae$ is the set
that characterizes the specification, the verification process consists in
showing that is possible to \emph{derive} the specified aspects in $\mathcal{I}$
from $\Formulae$. Each proof system define a particular approach to deal with
formulae and, ergo, it presents specific aspects dealing with different logics
in distinct scenarios. Furthermore, it is possible to combine proof methods with
the ultimate goal to benefit from each method's advantages. 

In the literature, there are several theorem provers for modal logics. In this
work, we focus on \ksp~\cite{Nalon2016}, a theorem prover for the basic
multimodal language \system{K}{n}{}, which implements the clausal resolution
method proposed in~\cite{nalon2015modal}. Clauses are labelled by the modal
level at which they occur, helping to restrict unnecessary applications of the
resolution inference rules. Several refinements and simplification techniques in
order to reduce the search space for a proof are implemented. To get the best
performance for a particular formula, or class of formulae, it is important to
choose the right strategy and optimizations. Currently, \ksp~leaves these
choices to the user, so we are interested in steps towards the implementation of
an ``auto-mode'' in which the prover makes choices on its own, based on an
analysis of the input.

\ksp~performs well if the set of propositional symbols are uniformly distributed
over the modal levels. However, when there is a high number of propositional
symbols in just one particular level, the performance deteriorates. One reason
is that the specific normal form we use always generates satisfiable sets of
propositional clauses (i.e.\ clauses without modal operators). As resolution
relies on saturation, this can be very time consuming. We are currently
investigating the use of other tools in order to speed up the saturation
process. For instance, \emph{Boolean Satisfiability Solvers} can often solve
hard structured problems with over a million variables and several million
constraints~\cite{satchapter}. We believe that we can take advantage of the
theoretical and practical efforts that have been directed in improving the
efficiency of such solvers. 

Our implementation, which is work in progress, uses a SAT solver based on clause
learning by conflict analysis. We feed this solver with the satisfiable sets of
clauses generated and, each time it identifies a conflict in these sets due to
unit propagation from some variable assignment, one or more new clauses are
learnt from the conflict analysis procedure, which analyses the structure of
unit propagation and decides which literals to include in the new
clauses~\cite{cdclchapter}. As mentioned before, as we already know that these
sets are satisfiable, we are not particularly interested in the model generated
by the SAT solver, but we believe that by carefully choosing the set of clauses
and making use of the learnt clauses generated by MiniSat we may be able to
reduce the time \ksp~spends during saturation. 

The rest of this work is organized as follows:
