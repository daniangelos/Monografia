%\section{}
This chapter introduces \system{K}{n}{}, a \emph{propositional modal logic
language}, semantically determined by an account of necessity and possibility.

A propositional modal language is the well known propositional language augmented
by a collection of \emph{modal operators}. In classical logic, propositions or
sentences are either evaluated to true or false, in any model. Propositional
logic and predicate logic, for instance, do not allow for any further
possibilities. However, in natural language, we often distinguish between
various modalities of truth, such as \emph{necessarily} true, \emph{known to be}
true, \emph{believed to be} true or yet true \emph{in some future}, for example.
Therefore, one may think that classical logics lacks expressivity in this sense. 

Modal logic adds operators to express one or more of these different modes of
truth. Different modalities define different languages. The key concept behind
these operators is that they allow us to reason over relations among contexts or
interpretations, an abstraction that here we think as \emph{possible worlds}.
The purpose of the modal operators is to permit the information that holds at
other worlds to be examined --- but, crucially, only at worlds visible from the
current one via an accessibility relation~\cite{blackburn2002modal}. Then,
evaluation of a modal formula depends on a set of possible worlds and the
accessibility relations defined for these worlds. It is possible to define
several accessibility relations between worlds, and different modal logics are
defined by different relations.

The modal language which is the focus of this work is the extension of the
classical propositional logic that adds the unary operators: $\nec{a}$ and
$\pos{a}$, whose reading are ``is necessary by the agent $a$'' and ``is
possible by the agent $a$'', respectively. This language, known as
\system{K}{n}{}, is characterized by the schema $\nec{a}(\varphi \then \psi) \then
(\nec{a} \varphi \then \nec{a} \psi)$ (axiom \system{K}{}{}), where $a \in \Agents = \{1,
\ldots, n\}$ and $\varphi, \psi$ are well-formed formulae. The addition of other
axioms defines different systems of modal logics and it imposes restrictions on
the class of models where formulae are valid~\cite{chellas:modal_logic}. 

Worlds and their accessibility relations define a structure known as
\emph{Kripke model}. The satisfiability and validity of a formula depend on this
structure. For example, given a Kripke model that contains a set of possible
worlds, a binary relation of accessibility between worlds and a valuation
function that maps in which worlds a proposition symbol holds, we say that a
formula $\nec{}p$ is satisfiable at some world $\st$ of this model, if the
valuation function establishes that $p$ is true at all worlds accessible from
\st.

It is now time to formally define the modal language we will be working with. The
syntax and semantics of \system{K}{n}{} are showed in Sections~\ref{syntax}
and~\ref{semantics}, respectively, and the definitions presented in these two
sections are adaptations from~\cite{journals/jal/NalonD07}.

\section{Syntax}
\label{syntax}

The language of \system{K}{n}{} is equivalent to its set of \emph{well-formed
formulae}, denoted by \wff, which is constructed from a denumerable set of
\emph{propositional symbols} $\Prop = \{p, q, r, \ldots\}$, the negation
symbol $\neg$, the disjunction symbol $\lor$ and the modal connectives
$\nec{a}$, that express the notion of necessity, for each index $a$
in a finite, non-empty fixed set of labels $\Agents = \{1, \ldots, n\}, n
\in \mathbb{N}$.

\begin{definition}
\label{def:wff}
    The set of well-formed formulae, \wff, is the least set such that:
    \begin{enumerate}
        \item $p \in \wff$, for all $p \in \Prop$
            \vspace{.2ex}
        \item if $\varphi, \psi \in \wff$, then so are $\neg \varphi, (\varphi
            \lor \psi)$ and $\nec{a} \varphi$, for each $a \in \Agents$
    \end{enumerate}
\end{definition}

Just as the familiar first-order existential and universal quantifiers are duals
to each other, that is, $\forall x\ \formula \iff \neg \exists x\ \neg \formula$, we have
the dual connectives $\pos{a}$ for necessity, which express possibility, and
they are defined by $\pos{a} \formula \stackrel{def} \neg \nec{a} \neg \formula$, for each
$\agent \in \Agents$. Other logic operators may be used as abbreviations.
In this work, we consider the usual ones:
\begin{itemize}
    \item $\varphi \wedge \psi \stackrel{def} \neg(\neg \varphi \lor \neg \psi)$ (conjuction)
    \item $\varphi \then \psi \stackrel{def} \neg \varphi \lor \psi$ (implication)
    \item $\varphi \iff \psi \stackrel{def} (\varphi \then \psi) \land (\psi \then \varphi)$ (equivalence)
    \item $\textbf{false} \stackrel{def} \varphi \wedge \neg \varphi$ (\emph{falsum})
    \item $ \textbf{true} \stackrel{def} \neg \textbf{false}$ (\emph{verum}) 
\end{itemize}

Parentheses may be omitted if the reading is not ambiguous.  When $n = 1$, we
often omit the index in the modal operators, i.e., we just write $\nec{}
\varphi$ (or `box' \formula) and $\pos{}\varphi$ (or `diamond' \formula), for a
well-formed formula $\varphi$. 

We define as \emph{literal} a propositional symbol $p \in \Prop$ or its negation $\neg
p$, and denote by \Literals~the set of all literals. A \emph{modal literal} is a
formula of the form $\nec{a} l$ or $\pos{a} l$, with $l \in \Literals$ and $a
\in \Agents$.

The following function definitions based on formulae' syntax will be helpful
when we introduce the proof system we make use in this work:

\begin{definition}
    The \emph{modal depth} of a formula, $mdepth : \wff \longrightarrow \Nat$,
    represents the maximal number of nesting modal operators in this formula.
    Inductively, we have:
    \begin{enumerate}
        \item $mdepth(p) = 0$ 
        \item $mdepth(\neg \formula) = mdepth(\formula)$
        \item $mdepth(\formula \lor \psi) = \max\{mdepth(\formula), mdepth(\psi)\}$
        \item $mdepth(\nec{a} \formula) = mdepth(\formula) + 1$
    \end{enumerate}
    With $p \in \Prop$ and $\formula, \psi \in \wff$.
\end{definition}

\begin{definition}
    The \emph{modal level} function, $ml : \wff \longrightarrow \Nat$,
    represents the maximal number of modal operators in which scope the formula
    occurs. 
\end{definition}

For instance, in $\nec{a}\pos{a} p$, $mdepth(p) = 0$ and $ml(p) = 2$.

\section{Semantics}
\label{semantics}

The semantics of \system{K}{n}{} is presented in terms of Kripke structures.

\begin{definition}
    A Kripke model for \Prop~and $\Agents = \{1, \ldots, n\}$ is given by the tuple 
    \begin{equation}
        \Model = (\St, \st_0, R_1, \ldots, R_n, \pi)
    \end{equation}
    where $\St$ is a non-empty set of possible worlds with a distinguinshed world
    $\st_0$, the root of \Model; each $R_a$, $a \in \Agents$, is a binary relation
    on $\St$, that is, $R_a \subseteq \St \times \St$, and $\pi: \St \times \Prop
    \longrightarrow \{false, true\}$ is the valuation function that associates
    to each world $\st \in \St$ a truth-assignment to propositional symbols.
\end{definition}

\emph{Satisfiability} and \emph{validity} of a formula is defined in terms of the \emph{satisfiability relation}.

\begin{definition}
\label{relsat}
    Let $\Model = (\St, \st_0, R_1, \ldots, R_n, \pi)$ be a Kripke model, $\st \in \St$ and $\varphi, \psi \in \wff$. The \emph{satisfiability relation}, denoted by 
    \sat{\Model}{\st}{\varphi}, between a world \st~and a formula $\varphi$,
    is inductively defined by:
    \begin{enumerate}
        \item \sat{\Model}{\st}{p} if, and only if, $\pi(\st, p) = \textbf{true}$, for all $p \in \Prop$;
        \item \sat{\Model}{\st}{\neg\varphi} if, and only if, $
            \langle \Model, \st \rangle \not \models \varphi$;
        \item \sat{\Model}{\st}{\varphi\lor\psi} if, and only if,
            \sat{\Model}{\st}{\varphi} or \sat{\Model}{\st}{\psi}
        \item \sat{\Model}{\st}{\nec{a} \varphi} if, and only if, for all $t\in
            \St$, $(\st, t) \in R_a$ implies  \sat{\Model}{t}{\varphi}
    \end{enumerate}
\end{definition}

Satisfiability is defined with respect to the root of a model. A formula $\varphi
\in \wff$ is said to be \emph{satisfiable} if there exists a Kripke model
$\Model = (\St, \st_0, R_1, \ldots, R_n, \pi)$ such that
\sat{\Model}{\st_0}{\varphi}. A formula is said to be \emph{valid} if it is
satisfiable in all models.

The satisfiability problem for \system{K}{n}{} corresponds to determining the
existence of a model in which a formula is satisfied. This problem is proven to be
PSPACE-complete~\cite{Spaan:coml}.

\begin{example}
    In Figure~\ref{example_semantics}.    
\end{example}

\begin{center}


\begin{figure}[tbh!]
\label{example_semantics}
\begin{center}
%\begin{tabular}{|cc|}
%\hline
%\begin{tikzpicture}[->,rectangle,draw,node distance=1.5cm,
                    %every state/.style={draw=green!30!black!50!,fill=green!10,shape=rectangle,rounded corners,drop shadow,minimum width=3cm}
                   %]
\begin{tikzpicture}[->,circle,draw,node distance=5cm
                   ]

    \node[circle,label=180:$\st_0$,draw,thick](w0){$
        \begin{array}{c}       
            p\\
        \end{array}
    $};
    \node[circle,label=180:$\st_1$,below right of=w0,draw,thick](w1) {$
        \begin{array}{c}       
            \neg p\\
        \end{array}
    $};
    \node[circle,label=0:$\st_2$,below left of=w0,draw,thick](w2) {$
        \begin{array}{c}       
            \neg p\\
        \end{array}
    $};
    \node[circle,label=0:$\st_3$,below left of=w1,draw,thick](w3) {$
        \begin{array}{c}       
             q\\
        \end{array}
    $};
    \path[thick] (w0) edge [below] node {2} (w1)
    (w1) edge [above,loop right] node {1} (w1)
    (w2) edge [above,loop left] node {1} (w2)
    (w0) edge [above,loop] node {1} (w0)
    (w1) edge [above] node {1} (w3)
    (w2) edge [above] node {1} (w3)
    (w0) edge [below] node {2} (w2);

\end{tikzpicture} 
\end{center}

%\hline
%\end{tabular}
    \caption{Example of a Kripke model for \system{K}{n}{}}
\end{figure}
\end{center}



\section{Normal Form}

Normal forms can provide elegant and constructive proofs of many standard
results. They can also provide proofs of results that are not readily proved by
standard means~\cite{fine1975}.

Formulae in \system{K}{n}{} can be transformed into a layered normal form called
\emph{Separated Normal Form for Normal Logics}~\cite{journals/jal/NalonD07}. 
