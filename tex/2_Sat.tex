The problem of determining whether a boolean formula is satisfiable has the
historical honor of being the first problem ever shown to be
NP-Complete~\cite{Cook}. Great theoretical and practical efforts are directed
in improving the efficiency of solvers for this problem, known as \emph{Boolean
Satisfiability Solvers}, or just \emph{SAT solvers}. Despite the worst-case
exponential run time of all the algorithms known, satisfiability solvers are
increasingly leaving their mark as a general purpose tool in the most diverse
areas~\cite{satchapter}. In essence, SAT solvers provide a generic combinatorial
reasoning and search tool. 

The underlying represenational formalism is propositional logic. A propositional
or Boolean formula is a logic expression defined over variables (or atomic
expressions), that take value in any binary set whose values oppose each other,
usually $\{\textbf{true}, \textbf{false}\}$ or just $\{0,1\}$, and boolean
connectives, any unary or binary function with one output, such as
$\land$ (conjunction), $\lor$ (disjunction), $\neg$ (negation), $\then$
(implication), $\iff$ (equivalence). Parentheses can be used to avoid ambiguous.
A \emph{truth assignment} to a set $V$ of Boolean variables is a map $\sigma : V
\rightarrow \{\textbf{true}, \textbf{false}\}$. A \emph{satisfying assignment}
for a formula $F$ is a truth assignment $\sigma$ such that $F$ evaluates to
\textbf{true}~\cite{satchapter}.

In the context of SAT solvers, we are interested in propositional formulae in
\emph{Conjuctive Normal Form} (CNF): $F$ is in CNF if it is a conjunction of
\emph{clauses}, where each clause is a disjunction of \emph{literals}, and each
literal is either a variable or its negation. For example, $F = (p \lor \neg q)
\land (\neg p \lor r \lor s) \land (q \lor r)$ is a CNF formula with four variables and
three clauses.

Therefore, the Boolean Satisfiability Problem (SAT) can be expressed as: Given a
CNF formula $F$, does $F$ have a satisfying assignment? One can be interested
not only in this decision problem, but also in finding an actual satisfying
assignment when it exists. All practical SAT solvers do produce such an
assignment if there exists one.

\section{The DPLL Procedure}
\label{sec:dpll}

\begin{algorithm}
    \SetKwInOut{Input}{Input}
    \SetKwInOut{Output}{Output}
    \Input{Two nonnegative integers $a$ and $b$}
    \Output{$\gcd(a,b)$}

    \underline{function Euclid} $(a,b)$

    \eIf{$b=0$}
    {
        return $a$\;
    }
    {
        return Euclid$(b,a\mod b)$\;
    }
\caption{Euclid's algorithm for finding the greatest common divisor of two nonnegative integers}
\end{algorithm}

\section{Conflict Driven Clause Learning}
\label{sec:cdcl}
