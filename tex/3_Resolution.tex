%\section{}
Resolution appeared in the early 1960s through investigations on performance
improvements of refutational procedures based on \emph{Herbrand Theorem}. In
particular, Prawitz' studies on such procedures brought back the idea of
unification. J.~A.~Robinson incorporated the concept of unification on a
refutation method, creating what was later known as resolution~\cite{casanova}. 

The standard rules for resolution systems take two or more premises with literals or
modal literals that are contradictory, and generate a resolvent. Most of these
systems work exclusively with clauses in a specific normal form. Resolution
systems are, in general, refutational systems, that is, to show that a
formula \formula~is valid, $\neg \formula$ is translated into a normal form. The
inference rules are applied until either no new resolvents can be generated or a
contradiction is obtained. The contradiction implies that $\neg \formula$ is
unsatisfiable and hence, that \formula~is valid.

%, for propositional
%logic, contains only one inference rule, which originates a new clause from two
%existing ones.

\section{Clausal Resolution}

Clausal resolution was proposed as a proof method for classical logic by
Robinson in 1965~\cite{Robinson65}, and was claimed to be suitable to be
performed by computer, as it has only one inference rule that is applied
many times. Nonclausal proof methods, in general, require a larger number of
rules, making implementation more difficult. Clausal resolution is a simple and
adaptable proof method for classical logics and, since it was proposed, a bank
of research into heuristics and strategies has been growing. 

%\subsection{Substitution}

%\subsection{Unification}

%\subsection{The resolution principle}
