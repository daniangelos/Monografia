%\section{}
Resolution appeared in the early 1960's through investigations on performance
improvements of refutational procedures based on \emph{Herbrand Theorem}. In
particular, Prawitz' studies on such procedures brought back the idea of
unification. J.~A.~Robinson incorporated the concept of unification on a
refutation method, creating what was later known as resolution~\cite{casanova}. 

The standard rule for resolution systems takes two or more premises with
literals that are contradictory, and generates a resolvent. Most of these systems
work exclusively with clauses in a specific normal form. Resolution systems are
refutational systems, that is, to show that a formula \formula~is valid, $\neg
\formula$ is translated into a normal form. The inference rules are applied
until either no new resolvents can be generated or a contradiction is obtained.
The contradiction implies that $\neg \formula$ is unsatisfiable and hence, that
\formula~is valid.

%, for propositional
%logic, contains only one inference rule, which originates a new clause from two
%existing ones.

\section{Clausal Resolution}

Clausal resolution was proposed as a proof method for classical logic by
Robinson in 1965~\cite{Robinson65}, and was claimed to be suitable to be
performed by computer, as it has only one inference rule that is applied
exhaustively. 

\subsection{Modal-Layered Resolution}

%Nonclausal proof methods, in general, require a larger number of rules, making
%implementation more difficult. Clausal resolution is a simple and adaptable
%proof method for classical logics and, since it was proposed, a bank of research
%into heuristics and strategies has been growing. 

%\subsection{Substitution}

%\subsection{Unification}

%\subsection{The resolution principle}

\section{Separated Normal Form with Modal Levels}

Formulae in \system{K}{n}{} can be transformed into a layered normal form called
\emph{Separated Normal Form with Modal Levels}, denoted by
\snf{$ml$}, proposed in~\cite{journals/jal/NalonD07}. A formula in \snf{$ml$} is a
conjunction of \emph{clauses} where the modal level in which they occur is
emphasized as a label.

We write $ml: \formula$ to denote that \formula~occurs at modal level $ml\in
\Nat \cup \{*\}$. By $*: \formula$ we mean that \formula~is true at
all modal levels. Formally, let $\wffml$ denote the set of formulae with
the modal level annotation, $ml : \formula$, such that $ml \in \Nat \cup \{*\}$
and $\formula \in \wff$. Let $\Model^* = (W, \st_0, R_1, \ldots, R_n, R_*, \pi)$
be a tree-like model and take $\formula \in \wff$. 

\begin{definition}
Satisfiability of labelled formulae is given by:

\begin{enumerate}
    \item $\Model^* \models ml : \formula$ if, and only if, for all worlds
        $\st \in W$ such that $depth(\st) = ml$, we have
        \sat{\Model^*}{\st}{\formula} 
    \item $\Model^* \models * : \formula$ if, and only if, $\Model^* \models
        \nec{*} \formula$
\end{enumerate}
    
\end{definition}

Observe that the labels in formulae work as a kind of \textit{weak} universal
operator, allowing us to reason about a set of formulae that are all satisfied
at a given modal level.

Clauses in \snf{$ml$} are defined as follows.

\begin{definition}
    Clauses in \snf{$ml$} are in one of the following forms:
    \begin{enumerate}
        \item Literal clause $\ \ \quad \qquad ml : \bigvee^r_{b=1} l_b$
        \item Positive $a$-clause $\ \qquad ml : l' \then \nec{a} l$
        \item Negative $a$-clause $\qquad ml : l' \then \pos{a} l$
    \end{enumerate}
    where $r, b \in \Nat, ml \in \Nat \cup \{*\}$ and $l, l', l_b \in
    \Literals$.
\end{definition}

Positive and negative $a$-clauses are together known as \emph{modal
$a$-clauses}, the index $a$ can be omitted if it is clear from the context.

The transformation of a formula $\formula \in \wff$ into \snf{$ml$} is achieved
by first transforming $\formula$ into its \emph{Negation Normal Form}, and then,
recursively applying rewriting and renaming~\cite{plaisted1986structure}.

\begin{definition}
    Let $\formula \in \wff$. We say that $\formula$ is in Negation Normal Form (NNF) if
    it contains only the operators $\neg, \lor, \land, \nec{a}$ and $\pos{a}$. Also,
    only propositions are allowed in the scope of negations.
\end{definition}

Let $\formula$ be a formula and $t$ a propositional symbol not occurring in
$\formula$. The translation of $\formula$ is given by $0 : t \land \rho(0 : t
\then \formula)$ -- for global satisfiability, the translation is given by $* :
t \land (\rho(* : t \then \formula)$ -- where $\rho$ is the \emph{translation
function} defined below. We refer to clauses of the form $0 : D$, for a
disjunction of literals $D$, as \emph{initial clauses}. 

\begin{definition}
    The translation function $\rho : \wffml \longrightarrow \wffml$ is defined
    as follows:
        \begin{align*}
            \rho(ml : t \then \formula \land \psi) & = \rho(ml : t \then \formula) \land \rho(ml : t \then \psi) \\
            \rho(ml : t \then \nec{a} \formula) & = (ml : t \then \nec{a} \formula) \text{, if \formula\ is a literal}\\
                                                & = (ml : t \then \nec{a} t') \land \rho(ml+1 : t' \then \formula) \text{, otherwise}\\
            \rho(ml : t \then \pos{a} \formula) & = (ml : t \then \pos{a} \formula) \text{, if \formula\ is a literal}\\
                                                & = (ml : t \then \pos{a} t') \land \rho(ml+1 : t' \then \formula) \text{, otherwise}\\
            \rho(ml : t \then \formula \lor \psi) & = (ml : \neg t \lor \formula
            \lor \psi) \text{, if $\psi$ is a disjunction of literals}\\
                                                  & = \rho(ml : t \then \formula \lor t') \land \rho(ml : t' \then \psi) \text{, otherwise}
        \end{align*}
        Where $t, t' \in \Literals$, $\formula, \psi \in \wff$, $ml \in
        \Nat \cup \{*\}$ and $r, b \in \Nat$.
\end{definition}

As the conjunction operator is commutative, associative and idempotent, we will
commonly refer to a formula in \snf{$ml$} as a set of clauses.

The next lemma, taken from~\cite{nalon2015modal}, shows that the transformation
into \snf{$ml$} preserves satisfiability.

\begin{lemma}
    Let $\formula \in \wff$ be a formula and let $t$ be a propositional symbol
    not occurring in $\formula$. Then: 
    \begin{enumerate}
        \item[$(i)$] $\formula$ is satisfiable if, and only if, $0 : t \land \rho(0 : t \then \formula)$ is satisfiable;
        \item[$(ii)$] $\formula$ is globally satisfiable if, and only if, $* : t \land \rho(* : t \then \formula)$ is satisfiable;
    \end{enumerate}
\end{lemma}

\begin{example}
    
\end{example}

\section{Modal-Layered Resolution Calculus for \system{K}{n}{}}

The motivation for the use of this labelled clausal normal form is that
inference rules can then be guided by the semantic information given by the
labels and applied to smaller sets of clauses, reducing the number of
unnecessary inferences, and therefore improving the efficiency of the proof
procedure~\cite{Nalon2016}. 

This calculus comprises a set of inference rules, given in Table~\ref{rules},
for dealing with propositional and modal reasoning. In the following, we denote
by $\sigma$ the result of unifying the labels in the premises for each rule.
Formally, unification is given by a function $ \sigma : \mathscr{P}(\Nat \cup
\{*\}) \longrightarrow \Nat \cup \{*\}$, where $\sigma (\{ml, *\}) = ml$ and
$\sigma (\{ml\}) = ml$, otherwise, $\sigma$ is undefined. The following
inference rules can only be applied if the unification of their labels is
defined (where $* - 1 = *$). Note that for GEN1 and GEN3, if the
modal clauses occur at the modal level $ml$, then the literal clause occurs at
the next modal level, $ml + 1$.

\begin{table}
    \caption{Inference rules}
\centering
{\scriptsize
\begin{tabular}{c}
\\
\begin{tabular}{cc}
%$
%\begin{array}{lrcl}
%\mbox{[IRES1]} &\universal(\constant{true} & \then & D \vee l)\\
 %&\universal(\constant{start} & \then  & D' \vee \neg l)\\ \cline{2-4}
 %&\universal(\constant{start} & \then & D \vee D')
%\end{array}
%$
%&
%$
%\begin{array}{lrcl} 
%\mbox{[IRES2]} &\universal(\constant{start} & \then &  D \vee l)\\
 %&\universal(\constant{start} & \then & D' \vee \neg l)\\  \cline{2-4}
 %&\universal(\constant{start} & \then & D \vee D')
%\end{array}
%$
%& 
$
\begin{array}{lrl}
\mbox{[LRES]} &ml_1: & D \lor l\\
 &ml_2: & D' \lor \neg l\\  \cline{2-3}
    &\sigma(\{ml_1,ml_2\}): & D \lor D'
\end{array}
$
    &
$
\begin{array}{lrl}
\mbox{[MRES]} &ml_1: & l_1   \then  \nec{a} l \\
              &ml_2: & l_2  \then  \neg \nec{a} l\\ \cline{2-3}
              &\sigma(\{ml_1,ml_2\}): & \neg l_1 \lor \neg l_2
\end{array}
$
\end{tabular}
\\
\\
\\
$
\begin{array}{lrl}
    \mbox{[GEN2]}& ml_1: & {l'}_1 \then \nec{a} l_1 \\
                 & ml_2: & {l'}_2 \then \nec{a} \neg l_1 \\
                 & ml_3: & {l'}_3 \then \pos{a} l_2 \\ \cline{2-3}
                 & \sigma(\{ml_1, ml_2, ml_3\}): &  \neg  {l'}_1 \lor  \neg {l'}_2 \lor \neg {l'}_3
\end{array} 
$
\\
\\
\\
{\setlength{\arraycolsep}{1pt}
\begin{tabular}{cc}
$
\begin{array}{lrl}
    \mbox{[GEN1]} & ml_1: & {l'}_1  \then  \nec{a}\neg l_1 \\
              & \qquad \vdots  \\
              & ml_m: & {l'}_m  \then  \nec{a}\neg l_m \\
              & ml_{m+1}: & l'  \then  \pos{a}\neg  l \\
              & ml_{m+2}:  & l_1 \lor \ldots \lor l_m \lor l \\  \cline{2-3}
              & ml:  & \neg {l'}_1 \lor \ldots \lor \neg {l'}_m \lor \neg l'
\end{array} 
$
&
$
\begin{array}{lrl}
    \mbox{[GEN3]} & ml_1 : & {l'}_1 \then  \nec{a}\neg l_1 \\
             & \qquad \vdots  \\
             & ml_m : & {l'}_m  \then  \nec{a}\neg l_m \\
             & ml_{m+1} : & l'  \then \pos{a}  l \\
             & ml_{m+2} : & l_1 \lor \ldots \lor l_m  \\  \cline{2-3}
             & ml : & \neg {l'}_1 \lor \ldots \lor \neg {l'}_m \lor \neg l'
 \end{array}
 $\\

where $ml = \sigma (\{ml_1, \ldots, ml_{m+1}, ml_{m+2} - 1\})$
    &
where $ml = \sigma (\{ml_1, \ldots, ml_{m+1}, ml_{m+2} - 1\})$
\end{tabular}}\\
\\

\end{tabular}}%
\label{rules}%
\end{table}



\section{\ksp}
