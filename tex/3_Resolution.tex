\section{}
Resolution appeared in the early 1960s through investigations on performance
improvements of refutational procedures based on Herbrand Theorem. In
particular, Prawitz' studies on such procedures brought back the idea of
unification. J.~A.~Robinson incorporated the concept of unification directly to the
method of refutation, creating what was later known as resolution. 

Nonclausal methods, in general, require a large number of resolution rules,
making implementation difficult.

\section{Clausal Resolution}

Clausal resolution was proposed as a proof method for classical logic by
Robinson in 1965~\cite{Robinson65}, and was claimed to be suitable to be performed
by computer, as it has only one rule of inference that may be applied many
times.
