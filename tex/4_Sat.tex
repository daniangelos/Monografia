%\section{}
The problem of determining whether a formula in classical propositional logic is
satisfiable has the historical honor of being the first problem ever shown to be
NP-Complete~\cite{Cook}. Great theoretical and practical efforts have been
directed in improving the efficiency of solvers for this problem, known as
\emph{Boolean Satisfiability Solvers}, or just \emph{SAT solvers}. Despite the
worst-case exponential run time of all the algorithms known, satisfiability
solvers are increasingly leaving their mark as a general purpose tool in the
most diverse areas~\cite{satchapter}. In essence, SAT solvers provide a generic
combinatorial reasoning and search platform. 

In the context of SAT solvers for propositional provers, the underlying
representational formalism is propositional logic~\cite{satchapter}. We are
interested in formulae in \emph{Conjunctive Normal Form} (CNF): $\formula$ is in
CNF if it is a conjunction of \emph{clauses}, where each clause is a disjunction
of literals. For example, $\formula = (p \lor \neg q) \land (\neg p \lor r \lor
s) \land (q \lor r)$ is a CNF formula with four literals and three clauses. 

%% satisfying assignment %%
A propositional formula $\formula$ takes value in the set $\{false, true\}$. A
\emph{truth assignment} (or just assignment) to a set of literals
$\Literals$ is a map $\sigma : \Literals \longrightarrow \{false, true\}$. A
\emph{satisfying assignment} for $\formula$ is an assignment $\sigma$ such
that $\formula$ evaluates to $true$ under $\sigma$.


Therefore, the \emph{Boolean Satisfiability Problem} (SAT) can be expressed as:
Given a CNF formula $\formula$, does $\formula$ have a satisfying assignment? One can be
interested not only in the answer of this decision problem, but also in finding
the actual assignment that satisfies the formula, when it exists. All practical
SAT solvers do produce such assignment~\cite{cormen}.

\section{The DPLL Procedure}
\label{sec:dpll}

A \emph{complete} solution method for the SAT problem is one that, given the
input formula $\formula$, either produces a satisfying assignment for $\formula$
or proves that it is unsatisfiable.  One of the most surprising aspects
of the relatively recent practical progress of SAT solvers is that the best
complete methods remain variants of a process introduced in the early 1960’s:
the DPLL procedure~\cite{satchapter}, which performs a backtrack search in the
space of partial truth assignments~\cite{DavisPutman:1960}. A key feature of
DPLL is efficient pruning of the search space based on falsified clauses. Since
its introduction, the main improvements to DPLL have been
smart branch selection heuristics, extensions like clause learning and
randomized restarts, and well-crafted data structures such as lazy
implementations and watched literals for fast unit propagation~\cite{satchapter}.

\begin{algorithm}[!ht]
\label{alg:dpll}
    \SetKwInOut{Input}{Input}
    \SetKwInOut{Output}{Output}
    \Input{A CNF formula $F$ and an initially empty partial assignment $\rho$}
    \Output{UNSAT, or an assignment satisfying $F$}

    %\underline{function Euclid} $(a,b)$

    $(F, \rho) \leftarrow$ UnitPropagate$(F,\rho)$

    \If{$F$ contains the empty clause}
    {\Return~UNSAT}
    \If{$F$ has no clauses left}
    {Output $\rho$\\
    \Return{SAT}
    }
    $l \leftarrow$ a literal not assigned by $\rho$

    \If{DPLL-recursive$(F|_l, \rho \cup \{l\}) = $ SAT} 
    {\Return{SAT}}
    \Return{DPLL-recursive$(F|_{\neg l}, \rho \cup \{\neg l\})$}
\caption{DPLL-recursive$(F, \rho$)}
\end{algorithm}

\section{Conflict Driven Clause Learning}
\label{sec:cdcl}

\begin{algorithm}[!ht]
\label{alg:cdcl}
    \SetKwInOut{Input}{Input}
    \SetKwInOut{Output}{Output}
    \Input{}
    \Output{}

    \If{UnitPropagate$(F, \nu) == $ CONFLICT}
    {\Return{UNSAT}}

    $dl \leftarrow 0$

    \While{\textbf{not} AllVariablesAssigned$(F, \nu)$}
    {$(x, \nu) \leftarrow$ PickBranchingVariable$(F,\nu)$}

    \Return{SAT}
    
    \caption{CDCL$(F, \nu)$}
\end{algorithm}

\section{MiniSat}
\label{sec:minisat}
Minisat~\cite{minisat}

