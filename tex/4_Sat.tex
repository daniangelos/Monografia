%\section{}
The problem of determining whether a formula in classical propositional logic is
satisfiable has the historical honor of being the first problem ever shown to be
NP-Complete~\cite{Cook}. Great theoretical and practical efforts have been
directed in improving the efficiency of solvers for this problem, known as
\emph{Boolean Satisfiability Solvers}, or just \emph{SAT solvers}. Despite the
worst-case exponential run time of all the algorithms known, satisfiability
solvers are increasingly leaving their mark as a general purpose tool in the
most diverse areas~\cite{satchapter}. In essence, SAT solvers provide a generic
combinatorial reasoning and search platform. Beyond that, the source code
of many implementations of such solvers is freely available and can be
used as a basis for the development of decision procedures for more expressive
logics~\cite{giunchiglia2002sat}.

In the context of SAT solvers for propositional provers, the underlying
representational formalism is propositional logic~\cite{satchapter}. We are
interested in formulae in \emph{Conjunctive Normal Form} (CNF): $\formula$ is in
CNF if it is a conjunction of \emph{clauses}, where each clause is a disjunction
of literals. For example, $\formula = (p \lor \neg q) \land (\neg p \lor r \lor
s) \land (q \lor r)$ is a CNF formula with four variables and three clauses. 
We use the symbol $\emptyset$ to denote the \emph{empty clause}, i.e., the
clause that contains no literals. A clause with only one literal is referred to
as a \emph{unit clause}, and a clause with two literals, as a \emph{binary
clause}.  When every clause of $\formula$ has $k$ literals, we refer to
$\formula$ as a $k$-CNF formula.  

%% satisfying assignment %%
A propositional formula $\formula$ takes value in the set $\{false, true\}$. A
\emph{truth assignment} (or just assignment) to a set of variables $\Prop$, is
the valuation function $\pi$ as defined in Definition~\ref{semantics}. As in
propositional logic we have a unit set as the set of possible worlds $\St$, we
can omit this set from the function signature and write $\pi : \Prop \longrightarrow
\{false,true\}$, for simplicity. A \emph{satisfying assignment} for $\formula$
is an assignment $\pi$ such that $\formula$ evaluates to $true$ under $\pi$.  A
\emph{partial assignment} for a formula $\formula$ is a truth assignment to a
subset of the variables in $\formula$. For a partial assignment $\rho$ for a CNF
formula $\formula$, $\formula|_\rho$ denotes the simplified formula obtained by
replacing the variables appearing in $\rho$ with their specified values, removing
all clauses with at least one $true$ literal, and deleting all occurrences of
$false$ literals from the remaining clauses~\cite{satchapter}.

Therefore, the \emph{Boolean Satisfiability Problem} (SAT) can be expressed as:
Given a CNF formula $\formula$, does $\formula$ have a satisfying assignment? If
this is the case, $\formula$ is said to be \emph{satisfiable}, otherwise,
$\formula$ is \emph{unsatisfiable}.  One can be interested not only in the
answer of this decision problem, but also in finding the actual assignment that
satisfies the formula, when it exists. All practical SAT solvers do produce such
assignment~\cite{cormen}. 

\section{The DPLL Procedure}
\label{sec:dpll}

A \emph{complete} solution method for the SAT problem is one that, given the
input formula $\formula$, either produces a satisfying assignment for $\formula$
or proves that it is unsatisfiable~\cite{satchapter}. One of the most surprising
aspects of the relatively recent practical progress of SAT solvers is that the
best complete methods remain variants of a process introduced in the early
1960’s: the Davis-Putnam-Logemann-Loveland, or DPLL,
procedure~\cite{DavisLongemannLoveland:1962}, which performs a backtrack search
in the space of partial truth assignments. A key feature of DPLL is efficient
pruning of the search space based on falsified clauses. Since its introduction,
the main improvements to DPLL have been smart branch selection heuristics,
extensions like clause learning and randomized restarts, and well-crafted data
structures such as lazy implementations and watched literals for fast unit
propagation~\cite{satchapter}.

Algorithm~\ref{alg:dpll}, DPLL-recursive$(\formula, \rho)$, sketches the basic
DPLL procedure on CNF formulae~\cite{DavisLongemannLoveland:1962}. The main idea
is to repeatedly select an unassigned literal $l$ in the input formula and
recursively search for a satisfying assignment for $\formula|_l$ and
$\formula|_{\neg l}$. The step where such an $l$ is chosen is called a
\emph{branching step}. Setting $l$ to $true$ or $false$ when making a recursive call
is referred to as \emph{decision}, and is associated with a decision level which
equals the recursion depth at that stage. The end of each recursive call, which
takes $\formula$ back to fewer assigned literals, is called the
\emph{backtracking step}.

\begin{algorithm}[htp]
    \SetAlgoLined\DontPrintSemicolon
    \SetKwFunction{proc}{UnitPropagate}
    $(\formula, \rho) \leftarrow$ \proc{$\formula,\rho$}

    \If{$\formula$ contains the empty clause}
    {\Return~UNSAT}
    \If{$\formula$ has no clauses left}
    {Output $\rho$\\
    \Return{SAT}
    }
    $l \leftarrow$ a literal not assigned by $\rho$

    \If{DPLL-recursive$(\formula|_l, \rho \cup \{l\}) = $ SAT} 
    {\Return{SAT}}
    \Return{DPLL-recursive$(\formula|_{\neg l}, \rho \cup \{\neg l\})$}

    \vspace{2mm}
    \setcounter{AlgoLine}{0}
    \SetKwProg{myproc}{sub}{}{}
    \myproc{\proc{$\formula, \rho$}}{%
    \While{$\formula$ contains no empty clause but has a unit clause $x$}
    {%
        $\formula \leftarrow \formula|_x$\\
        $\rho \leftarrow \rho \cup \{x\}$
    }
    \Return{$(\formula, \rho)$}
    \nl \KwRet\;}
    \caption{DPLL-recursive$(\formula, \rho$)}
    \label{alg:dpll}
\end{algorithm} 

A partial assignment $\rho$ is maintained during the search and output if the
formula turns out to be satisfiable. If $\formula |_\rho$ contains the empty
clause, the corresponding clause of $\formula$ from which it came is said to be
\emph{violated} by $\rho$. To increase efficiency, unit clauses are immediately
set to $true$ as outlined in Algorithm~\ref{alg:dpll}, this process is called
\emph{unit propagation}. The literals whose negation does not appear, called
\emph{pure literals}, are also set to $true$ as a preprocessing step and, in
some implementations, during the simplification process after every branch.

Variants of this algorithm form the most widely used family of complete
algorithms for the SAT problem. They are frequently implemented in an iterative
manner, resulting in significantly reduced memory usage. The efficiency of
state-of-the-art SAT solvers relies heavily on various features that have been
developed, analysed and tested over the last decade. These include fast unit
propagation using watched literals, deterministic and randomized restart
strategies, effective clause deletion mechanisms, smart static and dynamic
branching heuristics and learning mechanisms. This last one is more discussed in
the next section.

\section{Conflict-Driven Clause Learning}
\label{sec:cdcl}

One of the main reasons for the widespread use of SAT in many applications is
that solvers based on clause learning are so effective in
practice~\cite{satchapter}. The main idea is to cache ``causes of conflict'' as
learned clauses, and utilize this information to prune the search in a different
part of the search space encountered later. Since their inception in the
mid-90s, \emph{Conflict-Driven Clause Learning} (CDCL) SAT solvers have been
applied, in many cases with remarkable success, to a number of practical
applications~\cite{cdclchapter}. The organization of CDCL SAT solvers is
primarily inspired by the DPLL procedure.
 
Algorithm~\ref{alg:cdcl}.

\begin{algorithm}[!ht]
    \SetKwInOut{Input}{Input}
    \SetKwInOut{Output}{Output}
    \Input{}
    \Output{}

    \If{UnitPropagate$(F, \nu) == $ CONFLICT}
    {\Return{UNSAT}}

    $dl \leftarrow 0$

    \While{\textbf{not} AllVariablesAssigned$(F, \nu)$}
    {$(x, \nu) \leftarrow$ PickBranchingVariable$(F,\nu)$}

    \Return{SAT}
    \caption{CDCL$(F, \nu)$}
\label{alg:cdcl}
\end{algorithm}

\section{MiniSat and Glucose}
\label{sec:minisat}
MiniSat~\cite{minisat}

Glucose~\cite{glucose}
