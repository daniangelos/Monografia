%To implement a calculus for a logic, one needs a representation of formulae and
%operations corresponding to the inference rules. Somewhat more abstractly, one
%has to define a state transition system. The states represent (sets of) formulae
%with their interrelationships, providing information on the development of the
%derivations up to the respective point and on their possible continuations. The
%transitions model the changes to the states as inference rules are applied.
%Better state transition systems for the same calculus can be obtained by
%refining the states, for instance such that they indicate directly where
%inference rules can be or have been applied. More common improvements define
%additional transitions, which are not based on the rules of the calculus, but
%simplify the formulae or eliminate redundancies in the search space. The state
%transition systems described in this chapter are based on sets of clauses and on
%graph structures imposed on them.
