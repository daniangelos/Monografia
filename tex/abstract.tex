Modal logics have been widely studied in Computer Science for allowing the
characterization of complex systems that express notions in terms of knowledge,
belief etc. In this work, we focus on the theorem prover for the basic
multimodal language \system{K}{n}{}: \ksp, which implements a clausal resolution
method. Clauses are labelled by the modal level at which they occur, helping to
restrict unnecessary applications of the resolution inference rules.  
\ksp\ performs well if the set of propositional symbols are uniformly distributed
over the modal levels. However, when there is a high number of 
variables in just one particular level, the performance deteriorates. One reason
is that the specific normal form we use always generates satisfiable sets of
propositional clauses. As resolution relies on saturation, this can be very time
consuming. We are currently investigating the use of Boolean Satisfiability
Solvers (SAT solvers), since they are able to solve hard structured problems
with several variables and constraints. We believe that
we can take advantage of the efforts that have been directed in improving the
efficiency of such solvers. 
Our implementation, which is work in progress, uses a SAT solver based on clause
learning by conflict analysis. We feed this solver with the satisfiable sets of
clauses generated by translation and, each time it identifies a conflict, one or
more new clauses are learnt from the conflict analysis procedure. We believe
that by carefully choosing the set of clauses and making use of the learnt
clauses generated by the solver, we may be able to reduce the time \ksp\ spends
during saturation. 


