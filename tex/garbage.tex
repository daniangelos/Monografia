
%\begin{definition} %saturation
    %Let $\Clauses$ be a set of clauses. 
%\end{definition}

%\begin{definition} %frame
    %A set of literals which does not include a complementary pair of literals is
    %called a \emph{frame}. If $\pframe$ is a frame and $\Clauses$ is a set of
    %clauses, then $\pframe$ is a frame of $\Clauses$ if, for all $\clause$ in
    %$\Clauses$, $\clause$ contains a literal of $\pframe$.
%\end{definition}

%A set of clauses $\Clauses$ is satisfiable if there is a frame $\pframe$ of
%$\Clauses$; otherwise $\Clauses$ is unsatisfiable~\cite{Robinson65}. From the
%satisfiability, it follows that any set of clauses which contains $\cfalse$ is
%unsatisfiable, and that the empty set of clauses is satisfiable.

%\begin{equation}
%\label{eq:res}
 %\begin{array}{lc}
     %\mbox{[RES]} & \clause_i \lor l \quad \neg l \lor \clause_j \quad \clause_j \subseteq \clause_i\\ \cline{2-2}
     %& \clause_i 
%\end{array}
%\end{equation}
%where $\clause_i$ and $\clause_j$ are clauses, $l$ is a literal and $\clause_i
%\lor \clause_j$ is the resolvent.

%\section{Modal-Layered Resolution}

%Nonclausal proof methods, in general, require a larger number of rules, making
%implementation more difficult. Clausal resolution is a simple and adaptable
%proof method for classical logics and, since it was proposed, a bank of research
%into heuristics and strategies has been growing. 

%\subsection{Substitution}

%\subsection{Unification}

%\subsection{The resolution principle}


