\begin{center}


\begin{figure}[tbh!]
\begin{center}
%\begin{tabular}{|cc|}
%\hline
%\begin{tikzpicture}[->,rectangle,draw,node distance=1.5cm,
                    %every state/.style={draw=green!30!black!50!,fill=green!10,shape=rectangle,rounded corners,drop shadow,minimum width=3cm}
                   %]
\begin{tikzpicture}[->,circle,draw,node distance=5cm
                   ]

    \node[circle,label=180:$\st_0$,draw,thick](w0){$
        \begin{array}{c}       
            p\\
        \end{array}
    $};
    \node[circle,label=180:$\st_2$,below right of=w0,draw,thick](w1) {$
        \begin{array}{c}       
            \neg p\\
        \end{array}
    $};
    \node[circle,label=0:$\st_1$,below left of=w0,draw,thick](w2) {$
        \begin{array}{c}       
            \neg p\\
        \end{array}
    $};
    \path[thick] (w0) edge [below] node {} (w1)
    (w1) edge [above,loop right] node {} (w1)
    (w2) edge [above,loop left] node {} (w2)
    (w0) edge [below] node {} (w2);

\end{tikzpicture} 
\end{center}

%\hline
%\end{tabular}
    \caption{Example of a Kripke model for \system{K}{n}{}}
\label{example_semantics}
\end{figure}
\end{center}

