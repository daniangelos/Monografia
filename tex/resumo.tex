Lógicas modais têm sido estudadas na Ciência da Computação por permitirem a
caracterização de sistemas complexos, envolvendo noções tais como
conhecimento e crença, por exemplo. Neste trabalho, focamos no provador dej
teoremas para a lógica multimodal básica \system{K}{n}{}: \ksp, que implementa
um método baseado em resolução clausal. Cláusulas são rotuladas pelo nível modal
em que ocorrem, auxiliando a restringir aplicações desnecessárias das regras de
inferência.
\ksp\ apresenta uma boa performance se o conjunto de símbolos proposicionais
estão bem distribuídos entre os níveis modais. Entretanto, quando há um grande
número de variáveis em um nível específico, o tempo de execução aumenta. Uma das
razões é que a transformação para a forma normal utilizada sempre gera conjuntos
satisfatíveis de cláusulas proposicionais. Isto pode acarretar em uma
significativa perda de performance, uma vez que resolução é baseada em
saturação. Atualmente, estamos investigando o uso de provadores para o problema
de Satisfatibilidade Booleana, ou SAT \emph{solvers}, uma vez que já se
mostraram capazes de resolver eficientemente problemas difíceis. Acreditamos
poder utilizar as técnicas desenvolvidas depois de décadas de esforços práticos
e teóricos direcionados a estes provadores.
Nossa implementação, que é um trabalho em andamento, utiliza um SAT
\emph{solver} baseado em aprendizado de cláusula através de análise de conflito.
Damos como entrada a este solver, o conjunto satisfatível de cláusulas gerado
pela transformação e, toda vez que um conflito é identificado, uma ou mais
cláusulas são aprendidas pelo procedimento de análise de conflito. Acreditamos
que, cuidadosamente escolhendo o conjunto de cláusulas e usando as cláusulas
aprendidas geradas pelo \emph{solver}, podemos reduzir o tempo que \ksp\ gasta
durante a saturação.
